\subsubsection{Aplicaciones en el mercado}

Existe una gran variedad de aplicaciones y herramientas de Aprendizaje Máquina gratuitas y de pago en el mercado, de las cuales algunas están disponibles desde principio de siglo. Entre las más importantes que han sido guía para el desarrollo el presente proyecto son Orange, KNIME, WEKA, Google AI PlatformL, MLJar y Auger.ai.

\begin{APAitemize}
    \item Orange \parencite{Demsar2004} es una librería originalmente de C++ que incluye algoritmos de Aprendizaje de Máquina y Minería de datos. Esta librería es una colección de módulos basados en Python que funcionan alrededor de la librería principal, lo que permite crear algoritmos que son más comprensibles en Python que en C++, siempre y cuando el tiempo de ejecución no sea un problema. Todas las funciones de Orange lo convierte en un \textit{framework} basado en componentes, ideal para las labores relacionadas a la Inteligencia Artificial, que puede ser utilizado tanto por expertos, investigadores y estudiantes en un ambiente gráfico o en consola. A pesar de que principalmente es una librería, también se comporta como un grupo herramientas gráficas que utiliza los métodos disponibles con el fin de proveer mayor usabilidad para el usuario final.
    \item El desarrollo de KNIME \parencite{BCDG07} inició en Enero del 2004 por un grupo de ingenieros de software en University of Konstanz. Este software es una aplicación de acceso libre desarrollada en Java, la cual le permite a los usuarios visualizar y crear flujos de datos, ejecutar pasos del proceso a voluntad y proporcionar inspección de resultados y modelos, por medio de las vistas y \textit{widgets} que ofrece.
    \item El proyecto WEKA \parencite{Hall2009} se enfoca en proveer una colección de algoritmos de Aprendizaje de Máquina y Minería de datos, con el fin de ayudar a los investigadores y estudiantes por igual. Este software permite al usuario de realizar comparaciones rápidas entre diversos algoritmos y métodos con base a un conjunto de datos previamente suministrado. Por otra parte, el entorno de trabajo de esta aplicación incluye algoritmos de Clasificación, Regresión, Agrupamiento y selección de características, pero de igual forma es posible explorar y transformar la información previa al entrenamiento de un modelo de Aprendizaje de Máquina.
    \item AI Platform \parencite{googleML} hace parte de los servicios ofrecidos de Google, pero enfocados en Inteligencia Artificial. Hay tres productos ofrecidos por este servicio, los cuales son AutoML Vision, AutoML Natural y AutoML Tables, los cuales varían en precios y operaciones permitidas de acuerdo con la suscripción que tenga el usuario, por ejemplo: el servicio de AutoML Tables tiene un valor de 19.32 dolares por hora más los costos de implementación de modelos, predicción por lotes y predicción en línea. 
    \item El enfoque de MLJar \parencite{mljar2018} es permitir realizar el proceso de creación de modelos de Aprendizaje de Máquina con el mínimo esfuerzo posible. Este servicio puede ser accedido por medio de la API o utilizando los servicios web, los cuales si tienen un costo una vez agotados los créditos iniciales. Esta plataforma ofrece los servicios de Clasificación, Regresión, Agrupamiento y Redes Neuronales en un solo paquete, al igual que la extracción de características y obtención de datos relevantes.
    \item Auger \parencite{Auger} es una aplicación web gratuita que ofrece principalmente los servicios de Clasificación y Regresión. Ofrece una búsqueda rápida y precisa basada en la optimización bayesiana patentada por Auger. Por último, esta plataforma elige el optimizador apropiado en función de las características del conjunto de datos, con la premisa que los optimizadores propietarios superan a otros abiertos.
\end{APAitemize}

 