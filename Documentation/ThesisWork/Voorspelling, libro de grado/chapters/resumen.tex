\noindent{La falta de habilidades y conocimientos en un tema específico, como lo es en este caso el Aprendizaje de Máquina lleva a los estudiantes a cambiar su ruta de aprendizaje en la mayoría de los casos, o si en definitiva se continua por el camino de la Inteligencia Artificial, el mayor obstáculo en el camino son las aplicaciones que no satisfacen las necesidades de aprendizaje, y que además tienen costos elevados. Es por tales motivos que los estudiantes se ven obligados a utilizar esos medios hasta que se agote la prueba gratuita o tener que recurrir a medios tradicionales como crear código fuente para entrenar los modelos, opción poco factible debido a la falta de conocimientos que presentan los estudiantes en sus primeros años de formación profesional. Es por esos motivos que principalmente en el presente proyecto se plantea desarrollar una aplicación de escritorio para el sistema operativo Windows 10 x64 en el transcurso de 8 meses a partir de Agosto del 2020, con el fin de permitir a estudiantes en su primeros años de formación académica en carreras como ingeniería de software, sistemas, informática y computación, la selección del modelo, hiper-parámetros, entrenamiento y predicción con base al conjunto de datos suministrado, por medio de una interfaz de usuario sencilla de comprender y documentación detallada de las funciones incorporadas.}

\noindent{\textbf{Palabras clave: }Aprendizaje de Máquina Supervisado, Aplicación de escritorio, Python.}