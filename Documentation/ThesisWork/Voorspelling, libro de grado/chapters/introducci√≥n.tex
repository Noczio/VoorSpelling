En toda enseñanza el uso de herramientas juega un papel fundamental. A partir de esa idea se considera desarrollar el proyecto Voorspelling como una alternativa a las aplicaciones ya existentes, con el fin de facilitar la formación de estudiantes en Aprendizaje de Máquina Supervisado. Este proyecto busca enseñar el procedimiento necesario para crear un modelo de Aprendizaje de Máquina Supervisado, otorgando la posibilidad de realizar el proceso paso a paso o de forma automática a partir de las opciones disponibles para construir un modelo de Inteligencia Artificial. Los estudiantes dispuestos a recorrer la formación en Aprendizaje de Máquina suelen enfrentarse a más problemas de los que pueden prever en un inicio. Esto se debe a que los estudiantes en los primeros semestres de formación profesional en carreras como Ingeniería de Software, Sistemas, Informática y Computación, carecen de los conocimientos y la guía necesaria para la comprensión de temas relacionados a la Inteligencia Artificial. A pesar de que la información existe, es poco accesible en algunos casos y en otros la documentación se encuentra desactualizada, lo cual obstaculiza el proceso de aprendizaje y lleva a los estudiantes al punto de desistir.

Voorspelling es desarrollado por un equipo conformado por dos desarrolladores y un director de proyecto, utilizando la metodología Scrumban y un ciclo de desarrollo a partir de etapas compuestas por actividades estrechamente relacionadas con los objetivos específicos del proyecto. Esta aplicación es desarrollada para el sistema operativo Windows 10 de 64 bits, utilizando el lenguaje Python tanto para el \textit{front-end} como para el \textit{back-end}, el entorno de desarrollo integrado empleado es PyCharm y el \textit{framework} para crear la interfaz de usuario es Qt Designer. Con Voorspelling se espera que estudiantes universitarios puedan tener mejor compresión en la creación de modelos de Aprendizaje Máquina Supervisado, creen sus propios modelos, puedan adentrarse en la solución de problemas y los utilicen en futuros campos de trabajo. Adicionalmente, está aplicación aprovecha los recursos del equipo donde es ejecutado, puede crearse modelos que utilicen conjuntos de datos a nivel empresarial si se considera necesario, y por último, carece de las desventajas que presentan las aplicaciones web que ofrecen servicios afines.
