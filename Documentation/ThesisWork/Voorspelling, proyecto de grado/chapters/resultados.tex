El presente proyecto responde a la necesidad de herramientas académicas para iniciar a estudiantes de carreras afines a las ciencias de la computación en aprendizaje de máquina supervisado, por medio de un software de escritorio para el sistema operativo Windows 10,  documentación detallada y la utilización de una interfaz de usuario que facilita el proceso de creación de modelos tanto automáticos como convencionales.

Voorspelling durante su desarrollo, como todo proyecto de software, requirió de cambios en sus requerimientos y diseño para adaptarse a las necesidades y problemas que se presentan durante un desarrollo de software. Esta aplicación como se menciona anteriormente no es la excepción, por tal motivo, incluso después de su primer lanzamiento oficial, requiere de mejoras y actualizaciones para obtener un producto que responda a las necesidades de sus usuarios.

El estado actual del software cumple con los requerimientos aceptados por el director de proyecto, se realizan mejoras constantes en el producto y hay una ruta trazada para las futuras actualizaciones, la cual identifica como prioridad la mejora de la interfaz gráfica, soporte para otros sistemas operativos e idiomas.

Toda la documentación y código fuente se encuentra disponible en el repositorio del proyecto desde Marzo del 2021 para que futuros grupos de desarrollo continúen o creen sus propias versiones a partir de la arquitectura existente, ya sea utilizando únicamente el \textit{back-end} de la aplicación o el sistema completo.
 
